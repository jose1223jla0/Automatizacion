\documentclass[11pt,a4paper,openany]{report}
\usepackage[utf8]{inputenc}
\usepackage[T1]{fontenc}
\usepackage{amsmath}
\usepackage{amssymb}
\usepackage{graphicx}
\usepackage[top=2.5cm, bottom=2.5cm, left=2.5cm, right=2.5cm]{geometry}
\usepackage[spanish]{babel}
\author{Andrade oscco jose}
\renewcommand{\baselinestretch}{1.5}
\usepackage{float}
\usepackage{array}
\usepackage{url}
\usepackage{lipsum}
\usepackage{apacite}
\usepackage{natbib}
\usepackage{enumitem}
\usepackage{tocloft}
\usepackage{float}
\usepackage{array}
\usepackage{url}
\usepackage{lipsum}
\usepackage{apacite}
\usepackage{natbib}
\usepackage{enumitem}
\usepackage{tocloft}
\renewcommand{\cftsubsecleader}{\cftdotfill{\cftdotsep}} 
\bibliographystyle{apalike}
\bibliographystyle{apacite}
\usepackage{setspace}
\renewcommand{\cftsubsecleader}{\cftdotfill{\cftdotsep}} 
\usepackage{listings}
\usepackage{xcolor}

\lstdefinestyle{mypython}{
	language=Python,
	keywordstyle=\color{blue},
	stringstyle=\color{red},
	commentstyle=\color{green},
	basicstyle=\ttfamily\small,
	%numbers=left,
	numberstyle=\tiny\color{gray},
	%stepnumber=1,
	numbersep=10pt,
	%showspaces=false,
	showstringspaces=false,
	frame=single
}


\lstset{
	language=SQL,
	basicstyle=\linespread{1.2}\ttfamily,
	keywordstyle=\color{blue},
	commentstyle=\color{green},
	stringstyle=\color{red},
	showstringspaces=false,
	tabsize=2,
	breaklines=true
}
\lstset{
	literate=%
	{á}{{\'a}}1
	{é}{{\'e}}1
	{í}{{\'i}}1
	{ó}{{\'o}}1
	{ú}{{\'u}}1
	{ñ}{{\~n}}1
	{Ñ}{{\~N}}1
	{Á}{{\'A}}1
	{É}{{\'E}}1
}

\begin{document}
	
\begin{titlepage}
	\centering
	\vspace*{0cm}
	{\large\textbf{UNIVERSIDAD NACIONAL JOSÉ MARÍA ARGUEDAS}} \\
	\vspace*{0.5cm}
	{\large\textbf{FACULTAD DE INGENIERÍA}} \\
	\vspace*{0.5cm}
	{\large\ ESCUELA PROFESIONAL DE INGENIERÍA DE SISTEMAS } \\
	\vspace*{1.2cm}
	\includegraphics[width=0.2\textwidth]{unajma.png} \\
	\vspace{1cm}
	{\large \textbf{Docente del Curso:}}\\
	\vspace{0.2cm}	
	{\large MG Grecia Isabel Cabezas Moran}\\
	\vspace{0.2cm}
	{\large \textbf{Curso:}}\\
	\vspace{0.2cm}
	{\large Automatización}\\
	{\large \textbf{Tema:}}\\
	\vspace{0.2cm}
	{\large Sistema de Comunicación en Tiempo Real entre Arduino y Página Web para Monitoreo de Temperatura Ambiental}\\
	\vspace{1cm}
	{\large \textbf{Integrantes:}}\\
	\vspace{0.2cm}
	{\large José Luis Andrade Oscco}\\
	\vspace{0.2cm}
	{\large Miguel Angel Lizana Quispe}\\
	\vspace{0.2cm}
	{\large Wilmer Mario Ortiz Avendaño}\\
	\vspace{3cm}
	\textbf{{\large Andahuaylas-Perú}}	
\end{titlepage}
	

	
\chapter{Resumen}
	
	El proyecto consiste en diseñar y desarrollar un prototipo de comunicación en tiempo real entre un dispositivo Arduino y una página web para el monitoreo de la temperatura ambiental. Este sistema permitirá la recopilación, transmisión, almacenamiento y visualización de datos.
	
	Para lograr este objetivo, se implementará una interfaz de comunicación serial entre el Arduino y un computador utilizando la biblioteca PySerial. El Arduino estará configurado con un sensor de temperatura para obtener lecturas y enviarlas al computador a través del puerto serial. Un script en Python leerá estos datos del puerto serial y los almacenará en una base de datos previamente configurada para gestionar y organizar la información.
	
\chapter{Introducción}
\section{Contexto y justificación}
	
\section{Objetivos}
\subsection{Objetivo General}
	Diseñar un  sistema de comunicación en tiempo real entre un dispositivo Arduino y una página web.
\subsection{Objetivo Específicos}
\begin{itemize}
		\item Implementar una interfaz de comunicación serial entre el Arduino y un computador usando PySerial para la transmisión de datos de temperatura.
		\item Programar un script en Python que lea los datos del puerto serial y los almacene en una base de datos.
		\item Configurar una base de datos para almacenar y gestionar los datos de temperatura recibidos desde el Arduino.
\end{itemize}
	
	
\section{Alcance del proyecto} 
	
	\textbf{\begin{flushleft}
			Diseñar y Configurar el Hardware
	\end{flushleft}}
\begin{itemize}
	\item Selección de un sensor de temperatura compatible con Arduino.
	\item Integración y configuración del sensor con el Arduino para obtener lecturas precisas de temperatura.
\end{itemize}

	\textbf{\begin{flushleft}
			Implementar la Comunicación Serial
	\end{flushleft}}
\begin{itemize}
	\item Configuración de la biblioteca PySerial en el computador.
	\item Desarrollo de un script en Python para leer los datos enviados por el Arduino a través del puerto serial.	
\end{itemize}


\section{Hipótesis o preguntas de investigación}
	
\chapter{Marco Teórico}
\chapter{Metodología}

\section{Diseño del sistema}

\subsection{Materiales y herramientas}



	\bibliography{bibliografia}
\end{document}